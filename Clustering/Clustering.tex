\documentclass[dvipdfmx, 10pt]{jsarticle}
\usepackage{mathtools}
\usepackage[margin=20truemm]{geometry}
\usepackage{amssymb}
\usepackage{amsmath}
\usepackage{algorithm}
\usepackage{algpseudocode}
\usepackage{hyperref}
\usepackage{framed}

\title{\textbf{クラスタリング(Clustering)}}
\author{}
\date{}

\begin{document}

\maketitle

\section*{クラスタリング(Clustering)}
クラスタリングはデータをある関係性に基づいてグループ化する教師なし学習の1つ. グループ化されたデータの集合をクラスタと呼ぶ. 
関係性としては, ユークリッド距離やマンハッタン距離などの距離とコサイン類似度とJaccard係数などの類似度がよく使われる. 
コサイン類似度はデータの構造(方向)が似ているほど上昇し, Jaccard係数は共有する要素を持っているほど上昇する. 
距離によるクラスタリングでは, なるべく距離が近いデータを同じクラスタに遠いデータを異なるクラスタに所属させる. 
類似度によるクラスタリングでは, なるべく類似度が高いデータを同じクラスタに類似度が低いを異なるクラスタに所属させる. 

\begin{oframed}
\noindent
ユークリッド距離
\begin{align*}
    L_2(\mathbf{x}, \mathbf{y}) 
    &= \|\mathbf{x} - \mathbf{y}\|_2\\
    &= (\sum_{i=1}^{m} (x_i - y_i)^2)^{\frac{1}{2}}
\end{align*}

\noindent
マンハッタン距離
\begin{align*}
    L_1(\mathbf{x}, \mathbf{y}) 
    &= \|\mathbf{x} - \mathbf{y}\|_1\\
    &= \sum_{i=1}^{m} |x_i - y_i|
\end{align*}

\noindent
コサイン類似度
\begin{align*}
    cos(\mathbf{x}, \mathbf{y}) 
    &= \frac{\mathbf{x}^{\top}\mathbf{y}}{\|\mathbf{x}\|_2  \|\mathbf{y}\|_2}
\end{align*}

\noindent
Jaccard係数
\begin{align*}
    cos(\mathbf{x}, \mathbf{y}) 
    &= \frac{|X \cap Y|}{|X \cup Y|}\\ 
    &= \frac{\mathbf{x}^{\top}\mathbf{y}}{\|\mathbf{x}\|_2^2 + \|\mathbf{y}\|_2^2 - \mathbf{x}^{\top}\mathbf{y}}
\end{align*}

\end{oframed}

\section*{代表的な手法}
クラスタリングの代表的な手法として階層的クラスタリング, k-means法, k-medoids法, 
Density-Based Spatial Clustering of Applications with Noise (DBSCAN), スペクトラルクラスタリングが挙げられる. 

\subsection*{階層的クラスタリング}
階層的クラスタリングはクラスタを順次併合または分割しながらデータを階層的に統合または結合する手法である. 
デンドログラムを用いてクラスタの階層構造を視覚化できるため, 
クラスタの階層構造を理解したい場合やクラスタ数が事前に不明確な場合に用いられる. 
個々のデータをクラスタとし, 近いクラスタを順次統合するボトムアップ型の手法を凝集型クラスタリングと呼ぶ. 
また, 全体を1つのクラスタとし, 分割を繰り返すトップダウン型の手法を分割型クラスタリングと呼ぶ. 

課題はどの時点の分割がふさわしいかを判定するのはユーザが決める必要があることと, 大規模なデータに対しては計算負荷が高いこと. 

近いクラスタを測る指標として, 以下にクラスタ間の距離の代表的なものを示す. 

\begin{oframed}
    単リンク法(最短距離法)
    \begin{align*}
        d(C_i, C_j) = min\{d(u,v) | u \in C_i, v \in C_j\}
    \end{align*}

    完全リンク法(最長距離法)
    \begin{align*}
        d(C_i, C_j) = max\{d(u,v) | u \in C_i, v \in C_j\}
    \end{align*}

    グループ平均法
    \begin{align*}
        d(C_i, C_j) = \frac{1}{|C_i||C_j|} \sum_{u \in C_i} \sum_{v \in C_j} d(u,v) 
    \end{align*}

    ウォード法
    \begin{align*}
        & d(C_i, C_j) = q(C_i \cup C_j) - q(C_i) - q(C_j) \\
        & q(C_i) = \sum_{u \in C_i} \|\mathbf{x}_i - \mathbf{\mu}_i\|_2^2 \\
        & \mathbf{\mu}_i = \frac{1}{|C_i|} \sum_{u \in C_i} \mathbf{x}_u
    \end{align*}
\end{oframed}

\subsection*{k-means法}
k-means法は各クラスタの重心を代表点とし, データをその代表点に最も近いクラスタに割り当てる作業を収束するまで繰り返す手法である. 
代表点とクラスタ内のデータ点の距離の二乗の総和を最小にする問題. 
単純なクラスタリング問題やクラスタ数が明確な場合, 計算量を抑えたい場合に用いられる. 

課題は事前にクラスタ数を知っている必要があることと, 
クラスタの形状は円(球)状であることを仮定しているので複雑なデータや, 特定の方向に分散したデータを上手く分類できないこと. 
また, 結果は初期値に依存し最適なクラスタを出力する保証はない. 

\subsection*{k-medoids法}
k-medoids法はk-means法の代表点のとり方を重心の座標ではなく, 実際のデータ点を代表点とする手法. 
具体的にはその点以外のクラスタ内の点でまでの非類似度の総和が最小になる点(medoids)を代表点とする. 
medoidとデータ点の距離の総和(二乗の総和ではない)を最小化する問題. 
k-means法よりも外れ値に強い. また座標がわからなくても各点の距離が分かればクラスタリングできるので, 
集合での距離にもとづいてクラスタリングをしたいときに用いられる. 

課題はk-means法よりも遅いこと. 

\subsection*{Density-Based Spatial Clustering of Applications with Noise (DBSCAN)}
DBSCANはデータの密度を基準としてクラスタリングを行う手法である. 手順はまず適当な点を選択し, その点と一定距離にある点の数を数える. 
その点の数がしきい値以上ならクラスタにし, 未満ならノイズ点とする. この操作を繰り返す. 
すべての点に対してクラスタの割り振りやノイズの判断ができれば, DBSCANのアルゴリズムは終了する. パラメータとして距離のしきい値
最小のデータ点の数のしきい値(クラスタを形成するために必要な最小点数)を設定する必要がある. 
密度に基づいたクラスタリングやクラスタの形状が不規則な場合に用いられる. 
クラスタの数を設定する必要がないこと, 外れ値に強いことが利点である. 
また点同士の距離を判断基準とするためデータの分布より密度が優先されるので, データの分布に左右されない. 

課題は大規模なデータに対しては計算コストが高くなる傾向があること, 
密度がデータ間で異なる場合に距離のしきい値を設定するのが難しく精度が悪いこと, 
どのようなパラメータにするかよって結果が大きく変わるため, パラメータの設定に精度が左右されやすいこと. 
また高次元データでは, 距離尺度がスパースになるため密度を基準としたクラスタリングが不正確になる場合があるので, 
主成分分析(PCA)などで次元圧縮を行い, 対策をする必要がある. 

\subsection*{スペクトラルクラスタリング}
スペクトラルクラスタリングは重み付き無向グラフで表現される関係データをクラスタリングする手法である. 
クラスタ間にまたがる重みであるCutを正規化したRatioCutを最小化する問題となる. 
グラフの隣接行列と次数行列からなるラプラシアンの性質を利用することで, RatioCut最小化問題の緩和問題は固有値問題に定式化可能.  
よってラプラシアンの下位k個の固有値に対する固有ベクトルを求めているとも解釈できる. 
また, 固有ベクトルによりグラフ頂点をk次元のユークリッド空間に埋め込み, 
ユークリッド空間におけるクラスタリングアルゴリズムを実行しているとも解釈できる. 
スペクトラルクラスタリングでは連結性に注目してクラスタリングを行うため, k-means法などではうまくクラスタリングできなかった
円(球)状でないデータをうまくクラスタリングできる. 

課題はデータをグラフに変換する必要があること. 
変換方法としては, データ点同士がすべて結ばれる全結合法と, 距離が\(\epsilon\)以下のデータ点同士だけを結ぶ\(\epsilon\)近傍法と, 
各データ点について, 距離が近い上位\(k\)個のデータ点と結ぶ\(k\)近傍法などがある. 
全結合法の利点はデータの全体的な関係性を反映するため比較的安定した結果が得られること. 
しかし課題は大規模データセットでは計算コストが高くなること. 
\(\epsilon\)近傍法の利点はスパースな(疎な)グラフを生成できるため, 大規模データでも扱いやすいこと. 
またデータの局所的な構造に焦点を当てることができる. 
しかし課題はパラメータ\(\epsilon\)の選択が難しいこと. 値が小さすぎると孤立点が多くなり, 逆に大きすぎると全結合法に近くなる. 
\(k\)近傍法の利点は, \(\epsilon\)近傍法と同じくスパースな(疎な)グラフを生成できるため, 大規模データでも扱いやすいこと. 
さらに\(\epsilon\)近傍法よりもパラメータ選択が簡単である. 課題も\(\epsilon\)近傍法と同じく, 
パラメータ\(\epsilon\)の選択が難しいこと. 値が小さすぎると孤立点が多くなり, 逆に大きすぎると全結合法に近くなる. 

\section*{クラスタリングの評価指標}
クラスタリングの評価指標として, 最適なクラスタ数を決定するための方法であるエルボー法, 
各データ点が適切なクラスタに属しているかを評価するシルエットスコア, 
クラスタ間の分離と内部のコンパクトさを評価するダビース・ボルディン指数が挙げられる. 

\subsection*{エルボー法}
エルボー法はクラスタ内誤差平方和(SEE)値が小さいほど歪みのない(クラスタリングがうまくいっている)良いモデルという考え方から
最適なクラスタ数の検討を付ける手法. クラスタごとのSSE値をプロットした図を参照し, 
SSE値が"ヒジ"のようにガクンと曲がった点が最適なクラスタ数とみなす, つまりSSE値の低下が適切なクラスタ数を探す. 
しかし, 現実のデータを使ってエルボー図を書いた場合きれいなヒジが現れることは少ない. 他の評価指標と併用することが推奨される.

\subsection*{シルエットスコア}
シルエットスコアはクラスタ内の凝集度とクラスタ間の分離度を用いて計算されるクラスタリングの効果を測る指標の1つである. 
クラスタ間の距離が離れているほど, クラスタ内の密度が高いほど良いモデルという考え方からクラスタリングの効果を数値化する. 
凝集度は各データ点に対して, その点が属するクラスタ内の他の点との平均距離. 
これはデータ点がそのクラスタにどれだけフィットしているかを示す. 
凝集度が小さいほど, そのデータ点はクラスタ内でより密接している.
分離度は各データ点に対して, 最も近い別のクラスタとの平均距離. 
この距離は, 異なるクラスタとの分離を示す. 分離度が大きいほど, そのデータ点は他のクラスタから遠く離れている. 
シルエットスコアは\(-1\)から\(1\)の範囲で, \(1\)に近いほどクラスタリングの品質が高い. 
\(0\)に近いほどクラスタの境界が曖昧であることを示し, 負の値はクラスタリングが不適切である可能性がある. 

課題は高次元データでは計算コストが高くなる場合がある. 
またシルエットスコアは球状や均一な密度のクラスタを前提としているため, 
複雑な形状や異なる密度のクラスタには適用が難しい場合があるので, 
DBSCANやスペクトラルクラスタリングなど, 形状や密度に強い手法を併用する.

\begin{oframed}
    \noindent
    シルエットスコア aは凝集度, bは分離度を表す. 
    \begin{align*}
        s = \frac{b-a}{max(a, b)}  
    \end{align*}
\end{oframed}

\subsection*{ダビース・ボルディン指数}
ダビース・ボルディン指数は最もコンパクトでよく分離されたクラスタを生成したクラスタリングを特定するための指標の1つである. 
各クラスタについて, クラスタサイズの合計をクラスタ間距離で割った比率が最大となる代替クラスタを見つけ, 
その値をデータ中の全クラスタについて平均化したものである. \(k=2\)が理想的なクラスタサイズになりがち. 

\begin{oframed}
    \noindent
    ダビース・ボルディン指数
    \begin{align*}
        & D_i = max_{i \neq j}(\frac{|C_i|+|C_j|}{d(C_i, C_j)}) \\
        & DB = \frac{1}{n} \sum_{i=1}^{n} D_i
    \end{align*}
\end{oframed}

\section*{chatGPTによる試験対策問題}
クラスタリングは, 機械学習においてデータを特定の基準に基づいてグループ化する手法であり, 教師なし学習の一種である. クラスタリング手法には, K-means法や階層的クラスタリング, 密度に基づく手法(例: DBSCAN)などが含まれる. 

\subsection*{問1: K-meansクラスタリング}
\begin{enumerate}
    \item K-meansクラスタリングにおいて, 初期値として選択される「クラスタ中心点」の選択が結果にどのような影響を与えるか説明しなさい. 
    \item 「エルボー法(Elbow Method)」とは何か説明し, K-meansクラスタリングにおいてどのように活用されるか述べなさい. 
\end{enumerate}

(回答1)
k-means法はクラスタの中心の更新とクラスタの更新をクラスタの中心が変化しなくなるまで繰り返すことで, クラスタリングを行う. 
各データ点は一番近いクラスタの中心のクラスタに所属するので, 
初期値として選択されるクラスタ中心点は, アルゴリズムが収束する局所的な結果に強い影響を与える. 
よって適切でない初期値が選ばれると, クラスタリング結果が不適切になる可能性があり, 
初期値の選択に依存しない結果を得るためには, 複数回の初期値設定を試行して最適解を採用する方法が有効である.   

(回答2)
エルボー法は適切なクラスタ数を判断するための手法. 
クラスタ内のデータ点同士の距離の二乗和(Sum of Squared Errors, SSE)を評価指標とし, クラスタ数の変化とSEE値をプロットする. 
小さいほど密度が高く良いクラスタであるという考え方から, このプロットでSEE値が緩やかに低下する(ヒジのような部分)を探し, 
そのときのクラスタ数を最適なクラスタ数とする. すべてのデータに対してヒジが現れるわけではないので, 他の手法との併用が推奨される. 
k-means法では, クラスタ数(k)を固定してクラスタリングを行うため, 適切なクラスタ数を判定するために活用される. 

\subsection*{問2: 階層的クラスタリング}
\begin{enumerate}
    \item 階層的クラスタリングにおける「凝集型(Agglomerative)」と「分割型(Divisive)」の違いを説明しなさい. 
    \item デンドログラム(Dendrogram)とは何かを説明し, それがクラスタリング結果の解釈にどのように役立つか述べなさい. 
\end{enumerate}

(回答1)
凝集型は初期状態では各データ点を1つのクラスタ(シングルトン)として扱い, 
クラスタ間の距離が最も近いものを順次併合する. このプロセスをすべてのデータが1つのクラスタになるまで繰り返す. 
分割型は初期状態ではすべてのデータを1つの大きなクラスタとして扱い, 
クラスタ間の距離に基づいて分割を進める. 最終的に各データ点が独立したクラスタ(シングルトン)になるまで繰り返す. 

(回答2)
デンドログラムは, 階層的クラスタリングにおいてクラスタの分割と併合の過程を樹形図として可視化したもの. 
縦軸はクラスタ間の距離を表し, 横軸はデータポイントまたはクラスタを示す. デンドログラムを活用することで, 
クラスタリング結果の階層構造を視覚的に理解できるだけでなく, 適切なクラスタ数を決定する際に役立つ. 

\subsection*{問3: DBSCANクラスタリング}
\begin{enumerate}
    \item DBSCANの「コアポイント」「ボーダーポイント」「ノイズポイント」の定義を説明しなさい. 
    \item DBSCANがK-meansクラスタリングや階層的クラスタリングと比較して, どのような点で有利であるか説明しなさい. 
\end{enumerate}

(回答1)
DBSCANは密度を基準として行うクラスタリングの手法である. 
あるデータ点の一定の範囲内(\(\epsilon\)近傍)にあるデータ点の数がしきい値(最小データ点数)を超えたら, それらをクラスタに含める. 
超えなかったらノイズ点とする. この操作をすべてのデータ点がクラスタに所属するかノイズ点と判別されるまで繰り返す. 
\(\epsilon\)と最小データ点数はユーザが設定する必要がある. 
この時クラスタ含めるかどうか基準となるある点がコアポイント, 一定の範囲内に含まれる点をボーダーポイント, 
一定の範囲内に含まれずにノイズ点と判定された点をノイズポイントと呼ぶ. 

(回答2)
DBSCANが他の手法と比べて有利な点は, 
ユーザが設定する必要があるのは\(\epsilon\)と最小データ点数のみなので, クラスタ数を設定しなくてもクラスタリングが行えること. 
k-means法ではうまくクラスタリングできない円, 球状以外のデータに対してもクラスタリングが行えること. 
初期値に依存しないため, K-meansのような局所解に陥る可能性が低いこと. 
階層的クラスタリングに比べて大規模なデータセットでも計算コストが低いこと. 
またDBSCANは外れ値(ノイズ点)を自動的に検出し, クラスタに含めないためノイズが多いデータセットに対しても効果的である. 

\subsection*{問4: クラスタリング手法の評価}
\begin{enumerate}
    \item クラスタリングの評価指標として「シルエットスコア(Silhouette Score)」がある. この指標を用いることで, 
    クラスタリングの結果をどのように評価できるか説明しなさい. 
    \item クラスタリング結果を可視化する際に用いられる手法を2つ挙げ, それぞれの概要を説明しなさい. 
\end{enumerate}

(回答1)
シルエットスコアは分離度と凝集度をもとにしたクラスタリングの精度を測る指標. 
クラスタ間の距離は離れているほどよく, クラスタ内は密集しているほどよいクラスタリングという考え方からくる指標である. 
分離度はあるデータ点が自分が所属していないクラスタの中で一番近いクラスタとの平均距離で, どれだけ離れているかを表す. 
凝集度はあるデータ点が自分が所属しているクラスタの中で他の点との平均距離で, どれだけ密集しているかを表す. 
スコアが1に近いほどクラスタ間の分離が良好で, クラスタ内の密集度が高いを示す. 
0に近いほどクラスタ間の分離が不明瞭で, クラスタ構造が曖昧である. 
-1に近いほどデータ点が誤ったクラスタに割り当てられている可能性が高い. 
\(a\)は分離度, \(b\)凝集度とする. 
\begin{align*}
    s = \frac{b-a}{max(a, b)}
\end{align*}

(回答2)
クラスタリング結果を可視化することで, クラスタ間の分離やデータの分布を直感的に理解することができる. 
以下に代表的な手法を2つ挙げる:

\begin{enumerate}
    \item \textbf{主成分分析 (PCA: Principal Component Analysis)}

    主成分分析は, 高次元データを低次元(通常は2次元または3次元)に変換する手法である. この手法を用いることで, クラスタリング結果を平面や空間上にプロットし, 視覚的に確認することができる. PCAではデータの分散が最大になる方向(主成分)を基に次元削減を行う.
    \begin{itemize}
        \item \textbf{長所}: 高次元データを効率的に可視化できる.
        \item \textbf{短所}: 非線形なデータ構造を十分に表現できない場合がある.
    \end{itemize}

    \item \textbf{t-SNE (t-Distributed Stochastic Neighbor Embedding)}

    t-SNEは, 非線形次元削減手法であり, 高次元データを2次元または3次元空間に投影して可視化する. この手法はデータ点間の局所的な構造(近いデータ点同士の関係性)を保持しつつ, 次元を削減するため, クラスタリング結果の視覚的な理解を深めることができる.
    \begin{itemize}
        \item \textbf{長所}: 非線形構造を保持できるため, 複雑なデータの分布を表現するのに適している.
        \item \textbf{短所}: 計算コストが高く, 大規模データには不向きな場合がある.
    \end{itemize}
\end{enumerate}

\end{document}