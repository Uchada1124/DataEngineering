\documentclass[dvipdfmx, 10pt]{jsarticle}
\usepackage{mathtools}
\usepackage[margin=20truemm]{geometry}
\usepackage{amssymb}
\usepackage{amsmath}
\usepackage{algorithm}
\usepackage{algpseudocode}
\usepackage{hyperref}
\usepackage{framed}
\usepackage{booktabs}

\title{\textbf{因子分析(Factor Analysis)}}
\author{}
\date{}

\begin{document}

\maketitle

\section*{因子分析(Factor Analysis)}
多変量解析の手法で, 目的変数がなく説明変数が量的データの場合の手法は主成分分析と因子分析がある. 
どちらの手法も, 多くの変数から少数の潜在変数を生成することを目的としている. 類似している手法なので比較されることが多いが, 
主成分分析と因子分析では潜在変数の生成の仕方が異なる. 

主成分分析での潜在変数はデータの分散を最大限に説明する新しい変数で, 
必ず直交している必要がある. データの分散を最大限に説明するため主成分分析の潜在変数の1つとして総合力が含まれる. 
それ以外の潜在変数は相反する概念のものとなる. (例えば, 総合力と文系能力と理系能力など)

因子分析での潜在変数はデータの背後にある潜在的な共通因子(構造的な要因)を特定するための変数で, 直交する必要はない. 
また潜在的な共通因子を表現するため, 総合力は存在しない. 潜在変数は1つの概念となる. (例えば, 文系能力と理系能力など)

主成分分析は説明変数の個数が個体数より多くても実行できるが, 因子分析は説明変数の個数より個体数が多くなければならない. 

\section*{因子負荷量}
因子分析の出力は関係式の説明力, 関係式の係数, 各個体の得点となる. 
因子負荷量は, 観察変数と因子(潜在変数)との関連性の強さを示す値. 
因子負荷量の値が高いほど, その因子が該当する観察変数に強く影響を与えていることを示す. 
因子負荷量の値から各因子がどのような潜在変数であるかを判断する. 各個体の得点は因子得点と呼ばれる. 

\end{document}